\chapter{\PolyTeX\ tutorial} % (fold)
\label{cha:polytex_tutorial}

\noindent \textbf{Note:} \emph{This \PolyTeX\ tutorial is in development. It currently gives instructions for generating a \PolyTeX\ book and for converting from Markdown to \PolyTeX\@. Eventually I plan to include a full \PolyTeX\ tutorial, but for now I assume you either already know \LaTeX\ or are willing to use the listed resources to learn it.}

\section{\PolyTeX\ basics} % (fold)
\label{sec:polytex_basics}

\PolyTeX\ is a strict subset of the \LaTeX\ typesetting language. Here are some resources for getting started with \LaTeX:

\begin{itemize}
\item The generated example book (Section~\ref{sec:generating_a_polytex_book})
% \item The \href{https://github.com/softcover/softcover_book}{source of this book} (\href{https://github.com/softcover/softcover_book/blob/master/chapters/polytex_tutorial.tex}{this chapter} is written in \PolyTeX)
\item The \href{http://en.wikibooks.org/wiki/LaTeX}{Wikibooks LaTeX reference}
\item \href{https://www.google.com/search?q=latex+a+document+preparation+system+site:temple.edu}{The LaTeX manual}
\end{itemize}

\subsection{Generating a \PolyTeX\ book} % (fold)
\label{sec:generating_a_polytex_book}

You can generate an example \PolyTeX\ book as follows:

%= lang:console
\begin{code}
$ softcover new --polytex example_polytex_book
\end{code}

% subsection generating_a_polytex_book (end)

\subsection{From Markdown to \PolyTeX} % (fold)
\label{sec:markdown_to_polytex}

% subsection markdown_to_polytex (end)

If you've generated a Markdown book as in Section~\ref{sec:softcover_new} and want to switch to \PolyTeX, just copy the files in the \kode{generated\_polytex/} directory to the \kode{chapters/} directory and remove the Markdown files:

%= lang:console
\begin{code}
$ mv generated_polytex/*.tex chapters/
$ rm -f chapters/*.md
\end{code}

\noindent Then change \kode{Book.txt} to use \kode{*.tex} files in place of \kode{*.md}. Advanced users can remove \kode{Book.txt} and edit \kode{<book name>.tex} directly. (This requires running \kode{softcover build} once to bootstrap the system.) If you do this, you should add the main \LaTeX\ file to Git:

%= lang:console
\begin{code}
$ git add -f <book name>.tex
\end{code}

\subsection{\PolyTeX\ vs. \LaTeX} % (fold)
\label{sec:polytex_vs_latex}

To learn \PolyTeX, I recommend running a local server and trying out \LaTeX\ commands to see if they work:\footnote{Here we use the \kode{-p} flag to run the server on port 4001 in case another book is already running on port 4000 (the default).}

%= lang:console
\begin{code}
$ cd example_polytex_book
$ softcover server -p 4001
\end{code}

\noindent Visit \href{http://localhost:4001}{http://localhost:4001} to see the result. Then make changes, \href{http://www.urbandictionary.com/define.php?term=rinse%20repeat}{rinse}, \href{http://www.urbandictionary.com/define.php?term=rinse%20repeat}{and repeat}.


% subsection polytex vs_latex (end)

% section polytex basics (end)

\section{Included commands} % (fold)
\label{sec:included_commands}

Softcover includes some standard commands as part of \kode{softcover.sty}. The most common is \verb+\kode+ (so-called because \verb+\code+ was already taken), which sets text as inline code:

%= lang:latex
\begin{code}
Softcover includes some standard commands as part of \kode{softcover.sty}.
\end{code}

\noindent This is equivalent to Markdown's backtick environment:

%= lang:text
\begin{code}
Softcover includes some standard commands as part of `softcover.sty`.
\end{code}

\subsection{Colored text} % (fold)
\label{sec:colored_text}

Softcover defines the command \verb+\coloredtext+, used as follows:

%= lang:latex
\begin{code}
\coloredtext{<color name>}{<text>}
\end{code}

\noindent For example, \coloredtext{red}{this is red text}, and \coloredtext{teal}{this is teal text}. The \verb+\coloredtext+ command supports a huge number of colors, including
\coloredtext{red}{red},
\coloredtext{green}{green},
\coloredtext{blue}{blue},
\coloredtext{cyan}{cyan},
\coloredtext{magenta}{magenta},
\coloredtext{yellow}{yellow},
\coloredtext{black}{black},
\coloredtext{gray}{gray},
\coloredtext{white}{white} [white],
\coloredtext{darkgray}{darkgray},
\coloredtext{lightgray}{lightgray},
\coloredtext{brown}{brown},
\coloredtext{lime}{lime},
\coloredtext{olive}{olive},
\coloredtext{orange}{orange},
\coloredtext{pink}{pink},
\coloredtext{purple}{purple},
\coloredtext{teal}{teal},
\coloredtext{violet}{violet}, and all the \href{http://www.latextemplates.com/svgnames-colors}{svgnames colors} (such as \coloredtext{MediumSeaGreen}{medium sea green} and \coloredtext{CornflowerBlue}{cornflower blue}).

In addition to supporting a large number of named colors, for maximum flexibility Softcover defines the \verb+\coloredtexthtml+ command to allow for inclusion of \coloredtexthtml{E8AB3A}{arbitrary HTML colors} such as \coloredtexthtml{FF0000}{red text} using hexadecimal RGB values:

%= lang:latex
\begin{code}
\coloredtexthtml{E8AB3A}{arbitrary HTML colors} such as
\coloredtexthtml{FF0000}{red text}
\end{code}

\noindent \emph{Note}: Unlike HTML, \LaTeX\ requires six upper-case hex numbers, so

%= lang:latex
\begin{code}
\coloredtexthtml{ff0000}{red text}
\end{code}

\noindent and

%= lang:latex
\begin{code}
\coloredtexthtml{f00}{red text}
\end{code}

\noindent won't work.

% subsection colored_text (end)

\subsection{Code inclusion} % (fold)
\label{sec:polyex_code_inclusion}

Softcover supports code inclusion via a synatax similiar to the Softcover-flavored Markdown from Section~\ref{sec:code_inclusion}:

%= lang:latex
\begin{code}
%= <<(path/to/filename.ext)
\end{code}

\noindent This is equivalent to writing

%= lang:latex
\begin{code}
%= lang:ext
\begin{code}
<contents of filename.ext>
\end{code}
\end{code}

\noindent If the language can't be inferred from the extension, you can include a \kode{lang} option:


%= lang:latex
\begin{code}
%= <<(path/to/filename.ext, lang: ruby)
\end{code}

% subsection code_inclusion (end)



% section custom_commands (end)

% chapter polytex_documents (end)