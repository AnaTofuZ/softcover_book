\chapter{\PolyTeX\ tutorial} % (fold)
\label{cha:polytex_tutorial}

\begin{table}
\caption{The answer to LUE in several different bases.\label{table:answer}}
\begin{tabular}{|r|lc|}
  \hline
  2A & hexadecimal & (base 16) \\
  52 & octal & (base 8) \\
  101010 & binary & (base 2) \\
  \hline
  42 & decimal & (base 10) \\
  \hline
  \multicolumn{3}{|c|}{\textsc{All your base are belong to us.}} \\
  \hline
\end{tabular}
\end{table}


\noindent \textbf{Note:} This \PolyTeX\ tutorial is currently in development.

\PolyTeX\ is a strict subset of the \LaTeX\ typesetting language. You can generate an example \PolyTeX\ book as follows:

%= lang:console
\begin{code}
$ softcover new --polytex example_polytex_book
\end{code}

\noindent Then refer to the following resources:

\begin{itemize}
\item The generated example book
\item The \href{https://github.com/softcover/softcover_book}{source of this book} (\href{https://github.com/softcover/softcover_book/blob/master/chapters/polytex_tutorial.tex}{this chapter} is written in \PolyTeX)
\item The \href{http://en.wikibooks.org/wiki/LaTeX}{Wikibooks LaTeX reference}
\end{itemize}

Because not all of \LaTeX\ is supported by Softcover, to learn \PolyTeX\ I recommend running a local server and trying out commands to see if they work:\footnote{Here we use the \kode{-p} flag to run the server on port 4001 in case another book is already running on port 4000 (the default).}

%= lang:console
\begin{code}
$ cd example_polytex_book
$ softcover server -p 4001
\end{code}

\noindent Visit \href{http://localhost:4001}{http://localhost:4001} to see the result. Then make changes, \href{http://www.urbandictionary.com/define.php?term=rinse%20repeat}{rinse, and repeat}.

If you've generated a Markdown book as in Section~\ref{sec:softcover_new} and want to switch to \PolyTeX, just copy the files in the \kode{generated\_polytex/} directory to the \kode{chapters/} directory and remove the Markdown files:

%= lang:console
\begin{code}
$ mv generated_polytex/*.tex chapters/
$ rm -f chapters/*.md
\end{code}


% chapter polytex_documents (end)